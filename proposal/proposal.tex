\documentclass{article}
\usepackage{amsmath}
\usepackage[]{listings}
\usepackage{color}
\usepackage{courier}
\usepackage{graphicx}
\usepackage{caption}
\usepackage{subcaption}
\usepackage[margin=1.5in]{geometry}
\title{Machine Learning 10-601,Fall 2015, Project Proposal}
\author{Jai Prakash (jprakash), Utkarsh Sinha (usinha)}

\definecolor{mygreen}{RGB}{28,172,0} % color values Red, Green, Blue
\definecolor{mylilas}{RGB}{170,55,241}

\lstset{basicstyle=\footnotesize\ttfamily,breaklines=true}
\lstset{framextopmargin=50pt,frame=bottomline}

\begin{document}
\lstset{language=Matlab,%
    %basicstyle=\color{red},
    breaklines=true,%
    morekeywords={matlab2tikz},
    keywordstyle=\color{blue},%
    morekeywords=[2]{1}, keywordstyle=[2]{\color{black}},
    identifierstyle=\color{black},%
    stringstyle=\color{mylilas},
    commentstyle=\color{mygreen},%
    showstringspaces=false,%without this there will be a symbol in the places where there is a space
    numbers=left,%
    numberstyle={\tiny \color{black}},% size of the numbers
    numbersep=9pt, % this defines how far the numbers are from the text
    emph=[1]{for,end,break},emphstyle=[1]\color{red}, %some words to emphasise
    %emph=[2]{word1,word2}, emphstyle=[2]{style},    
}


  \maketitle
  The CIFAR-10 dataset consists of images and labels from one of 10 classes. Our plan is to survey and implement a classification algorithm using three different approaches. Firstly, we intend to look at a simple \textbf{bag-of-words} algorithm to classify the images. Given that we have 10 image classes, we hope to achieve an accuracy greater than 10\% using this. Given that these images are relatively small in size, it might be possible to use the entire image to generate the vocabulary of words. We intend to use SIFT\cite{sift} , GIST\cite{gabor}  or similar feature descriptors to generate the dictionary.\\

  Secondly, we want to use an \textbf{SVM} for classification based on features generated by using PCA. The reduced dimension features are used to train the SVM. This approach would use moment invariants of images to reduce the dimensionality of images\cite{svm}. The third approach we intend to use is a \textbf{sum-product neural network}\cite{neural}. This paper describes a discriminative training algorithm for such networks and also proposes a fix for the “diffusion problem” faced by standard gradient descent. The paper claims to achieve good results with the CIFAR-10 dataset as well. Training this network uses a backpropagation style algorithm to compute the gradient of the conditional log likelihood.\\

  We intend to generate a larger training dataset by transforming the original images. This can be achieved by rotating, mirroring and flipping the image about the axes. With a larger training dataset, our hypothesis is that the classifier will achieve better results that simply using the original dataset.\\

    \begin{thebibliography}{9}
      \bibitem{sift} David Lowe, et al “Distinctive Image Features from Scale-Invariant Keypoints”, IJCV, 2004
      \bibitem{gabor} Christian Saigian, et al. “Rapid Biologically-Inspired Scene Classification Using Features Shared with Visual Attention”, Pattern Analysis and Machine Intelligence, 2007
      \bibitem{svm} R.Muralidharan, et al. “Object Recognition Using Support Vector Machine Augmented by RST Invariants”, IJCSI, 2011
      \bibitem{neural} R Gens, et al. “Discriminative learning of sum-product networks”, Washington University, 2012
    \end{thebibliography}

\end{document}

